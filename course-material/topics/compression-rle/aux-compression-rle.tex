\usepackage{siunitx}
\usepackage{ifthen}

\usetikzlibrary{positioning}

\makeatletter
\def\light@caption@on{on}
\def\light@caption@off{off}
\def\light@caption@neutral{}
\makeatother

\tikzset{
  light/.pic={\draw node[circle,#1,minimum size=7mm,font=\tiny] {\csname light@caption@#1\endcsname};},
  light/.style={draw},
  on/.style={light,fill=yellow},
  off/.style={light,fill=gray!50},
  neutral/.style={light}
}

\title{Data Compression: RLE}
\author{Fr\'ed\'eric Vogels}

\begin{document}

\begin{frame}
  \titlepage
\end{frame}

\begin{frame}
  \frametitle{Run Length Encoding}
  \begin{itemize}
    \item RLE is very simple compression algorithm
    \item In order to compress, it must make some assumptions about input data
    \item The \texttt{GQU} compression algorithm assumed certain pairs appear more often than others
    \item RLE assumes that long runs of the same bytes occur often
    \item RLE is used in some image formats: drawn images do have long stretches of the same colour
          (TGA, PCX, fax machines)
  \end{itemize}
\end{frame}

\begin{frame}
  \frametitle{Run Length Encoding}
  \begin{center}
    \tt aaaaabbbbccccccccaaaaaaaaddd
  \end{center}
  \begin{itemize}
    \item How to make use of this structure to compress the data?
    \item Why not encode each series
         \[
           \underbrace{\texttt{xxx\dots x}}_{n \; \times}
         \]
         as
         \[
           n\texttt{x}
         \]
    \item This yields
          \begin{center}
            \tt 5a4b8c8a3d
          \end{center}
  \end{itemize}
\end{frame}

\begin{frame}
  \frametitle{Compressing Lonely Bytes}
  \begin{center}
    abcd
  \end{center}
  \begin{itemize}
    \item How to encode this using RLE?
    \item Can't we just compress this to \texttt{abcd}?
    \item This is rather dubious: it would mean that there is no input data
          that would be ``compressed'' to something larger
    \item RLE would be usable on all data: in the worst case, the compressed
          form would be as large as the original data
    \item This breaks a fundamental principe of compression
    \item What's the problem? Did we break maths?
  \end{itemize}
\end{frame}

\begin{frame}
  \frametitle{Compressing Lonely Bytes}
  \begin{center}
    abcd
  \end{center}
  \begin{itemize}
    \item Text must be encoded using ASCII/Unicode/\dots
          \begin{center} \small
            \begin{tabular}{cc}
              \textbf{character} & \textbf{ASCII code} \\
              \toprule
              \tt a & 97 \\
              \tt b & 98 \\
              \tt c & 99 \\
              \tt d & 100 \\
            \end{tabular}
          \end{center}
    \item {\tt abcd} is represented by the bytes
          \begin{center}
            97 98 99 100
          \end{center}
    \item The decompression algorithm will interpret this as
          \begin{itemize}
            \item Repeat \texttt{b} 97 times
            \item Repeat \texttt{d} 99 times
          \end{itemize}
  \end{itemize}
\end{frame}

\begin{frame}
  \frametitle{Compressing Lonely Bytes}
  \begin{itemize}
    \item Decompressor cannot make difference between
          \begin{itemize}
            \item Byte representing how many times next byte needs to be repeated
            \item Byte representing actual data
          \end{itemize}
    \item Decompressor expects that compressed data
          is structured as pairs of bytes $n,B$, meaning
          ``repeat $B$ $n$ times''.
    \item \texttt{abcd} needs to be compressed to
          \begin{center}
            \tt 1a1b1c1d
          \end{center}
          or, in bytes (decimal)
          \begin{center}
            \tt 1 97 1 98 1 99 1 100
          \end{center}
          or, in bytes (hexadecimal)
          \begin{center}
            \tt 01 61 01 62 01 63 01 64
          \end{center}
  \end{itemize}
\end{frame}

\begin{frame}
  \frametitle{Shortcomings of RLE}
  \begin{itemize}
    \item We found the Achilles heel of RLE
    \item It can compress runs of bytes well
    \item It utterly fails on solitary bytes
    \item Maths were right all along
    \item Huzzah for maths!
  \end{itemize}
\end{frame}

\begin{frame}
  \frametitle{Variations}
  \begin{itemize}
    \item Many variations possible
    \item For example, how to encode $n$
    \item Currently, we encode $n$ as a single byte
    \item Two problems
          \begin{itemize}
            \item $n = 0$ is useless: inefficient
            \item $n$ can't be greater than 255: longer series to be split up
          \end{itemize}
  \end{itemize}
\end{frame}

\begin{frame}
  \frametitle{Variations: Dealing with $n = 0$}
  \begin{itemize}
    \item $n = 0$ is useless
    \item We waste a perfectly good code
    \item We can adapt the algorithm so as to make $n = 0$ useful
    \item One possibility: interpret $n = 0$ as $n = 256$, so that
          you can support runs of 256 long
    \item Other possibility: use $n = 0$ to deal with solitary bytes by
          saying that $n = 0$ means that the following two bytes are solitary
  \end{itemize}
\end{frame}

\begin{frame}
  \frametitle{Variations: Dealing with $n$'s Maximum Value}
  \begin{itemize}
    \item We could use more bytes for $n$
    \item 1 byte: $n$ can go up to 255
    \item 2 bytes: $n$ can go up to 65535
    \item 4 bytes: $n$ can go up to $2^{32}-1$
    \item Is this interesting?
    \item Using 1: good
          when there are many short runs
    \item Using 4: good
          when there are at least a few very long runs
  \end{itemize}
\end{frame}

\begin{frame}
  \frametitle{Variations: Dealing with $n$'s Maximum Value}
  \begin{itemize}
    \item We could use variable length $n$
    \item $n = 255$ could mean that the next byte should be interpreted as a count-byte and added to the current $n$
    \item For example, consider text
          \[
            \underbrace{\texttt{aaa\dots a}}_{50\times}\underbrace{\texttt{bbb\dots b}}_{(260)\times}
          \]
    \item In our original encoding, this would become (in decimal)
          \[
            \underbrace{\tt 50 \quad 97}_{50\times\tt a} \quad \underbrace{\tt 255 \quad 98}_{255\times\tt b} \quad \underbrace{\tt 5 \quad 98}_{5\times\tt b}
          \]
    \item In our variable length $n$ encoding, we would get
          \[
            \underbrace{\tt 50 \quad 97}_{50\times\tt a} \quad \underbrace{\tt 255 \quad 5 \quad 98}_{(255+5)\times\tt b}
          \]
  \end{itemize}
\end{frame}

\begin{frame}
  \frametitle{Variations: Variable Length $n$}
  \structure{MIDI and UT8}
  \begin{itemize}
    \item Support integers of arbitrary length
    \item Bits of the integers are split in groups of 7 bits
    \item Each encoded byte contains a first group of 7 bit. The eighth bit is used to indicate whether
          the following byte also belongs to the same integer
  \end{itemize}
  \begin{overprint}
    \onslide<handout:1|1>
    \structure{Encoding 100}
    \begin{itemize}
      \item 100 in binary: \texttt{1100100}
      \item 7 bits needed
      \item Only one byte needed: \texttt{{\color{red} 0}\kern1pt {\bfseries 1100100}}
      \item {\color{red}\tt 0} means ``integer ends here''
    \end{itemize}

    \onslide<handout:2|2>
    \structure{Encoding 1000}
    \begin{itemize}
      \item 1000 in binary: \texttt{111\kern2pt 1101000}
      \item 10 bits needed
      \item Two bytes needed: \texttt{{\color{red} 1}\kern1pt {\bfseries 0000111} {\color{red} 0}\kern1pt {\bfseries 1101000}}
      \item {\color{red}\tt 1} means ``integer continues in next byte''
    \end{itemize}

    \onslide<handout:3|3>
    \structure{Encoding 100000}
    \begin{itemize}
      \item 100000 in binary: \texttt{110\kern2pt 0001101\kern2pt 0100000}
      \item 17 bits needed
      \item Three bytes needed: \texttt{{\color{red} 1}\kern1pt {\bfseries 0000110} {\color{red} 1}\kern1pt {\bfseries 0001101} {\color{red} 0}\kern1pt {\bfseries 0100000}}
    \end{itemize}
  \end{overprint}
\end{frame}

\end{document}


%%% Local Variables:
%%% mode: latex
%%% TeX-master: "compression-rle"
%%% End:
